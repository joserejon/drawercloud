\chapter{Conclusiones y trabajos futuros}

\section{Conclusiones}
Tras la finalización del proyecto se ha obtenido un prototipo funcional de la aplicación web, que cumple con casi todas las expectativas y los objetivos marcados desde el principio, además de una completa memoria del proyecto. \\
 
Gracias a las pruebas realizadas con usuarios podemos determinar que...... \\

En definitiva tenemos que \textbf{drawercloud} cumple con los requisitos para efectuar su función principal de alojar archivos en la nube, a excepción del espacio disponible de almacenamiento actual que es insuficiente para una aplicación de estas características. Esto se debe a que el espacio que nos ofrece Amazon es solamente de 2GB, debiéndose esto a que se trata de la versión gratuita. No obstante esto tendría solución comprando más espacio de almacenamiento y ofreciendo a los clientes distintas versiones, las cuales serían de pago para grandes cantidades de espacio y una gratuita que ofrecería menos capacidad. \\

\subsection{Conocimientos adquiridos}
Gracias al análisis y al desarrollo de este proyecto se han adquirido conocimientos en diversos ámbitos del desarrollo de software, en su mayoría en el sector web. Entre los conocimientos adquiridos podemos nombrar: 

\begin{itemize}
	\item Se ha profundizado mucho en el uso del lenguaje de programación \textbf{Python}, así como en el framework para desarrollo web \textbf{Django}, proporcionando así conocimientos de la arquitectura \textbf{MVC}.
	\item Técnicas para el desarrollo de interfaces web gracias a tecnologías como \textbf{Bootstrap} \textbf{jQuery} y \textbf{AJAX}.
	\item Realizar \textbf{pruebas con usuarios} para permitir encontrar errores o descubrir funcionalidades que podamos aplicar para hacer una aplicación más completa y competitiva.
	\item Aprender a realizar un correcto \textbf{desarrollo de la documentación}.
\end{itemize}

\section{Trabajos futuros}
drawercloud ofrece la funcionalidad necesaria para ser una herramienta de almacenamiento de archivos en la nube, pero para ello aún se deben pulir algunos aspectos y convertirla en un serio competidor. Para ello se deberían tener en cuenta algunas evoluciones en el software como las siguientes:

\begin{itemize}
	\item Comprar espacio de almacenamiento.Esta opción sería la maś importante ya que no disponemos de espacio suficiente para competir con otras aplicaciones de este tipo.
	\item Permitir subir \textbf{varios archivos} a la vez.
	\item Incluir las opciones de \textbf{arrastrar para mover} y \textbf{seleccionar} varios archivos.
	\item Añadir un \textbf{visor} de imágenes y \textbf{reproductores} de audio y vídeo más estéticos y con mayor funcionalidad.
	\item Perfeccionar aspectos de \textbf{rendimiento}	a la hora de subir archivos.
	\item Crear una versón de la aplicación para \textbf{móvil} y para \textbf{escritorio}.
\end{itemize}
