\chapter*{}
%\thispagestyle{empty}
%\cleardoublepage

%\thispagestyle{empty}

\input{portada/portada_2}

\clearpage
\thispagestyle{empty}

\begin{center}
{\large\bfseries drawercloud: aplicación web orientada al almacenamiento de archivos en la nube}\\
\end{center}
\begin{center}
José Manuel Rejón Santiago (alumno)\\
\end{center}

%\vspace{0.7cm}
\noindent{\textbf{Palabras clave}: aplicación web, cloud, pc, tablet, smartphone, servidor web, framework, responsive design}\\

\vspace{0.5cm}
\noindent{\textbf{Resumen}}\\

\textbf{drawercloud} es una aplicación web para el alojamiento de archivos en la nube (cloud). Los usuarios podrán realizar diversas tareas sobre sus archivos, como por ejemplo: Añadir/Eliminar archivos en la nube, reproducir archivos multimedia, organizar archivos en directorios, compartir archivos, descargar en un dispositivo (pc, tablet, smartphone)... De este modo, conseguiremos tener nuestros archivos almacenados en un servidor web, ahorrando espacio en nuestros dispositivos y con la seguridad de tener nuestro material a salvo en otro lugar.\\

Para el desarrollo de la web usaremos \textbf{Django} \cite{cita1}, que es un framework para aplicaciones web gratuito y open source, escrito en \textbf{Python} \cite{cita2}. La aplicación web contará con un \textbf{responsive design} para ser adaptada a otros dispositivos (como smartphones o tablets). En el lado del servidor, se usará la tecnología proporcionada por Amazon Web Services o Google.\\

Para el aspecto gráfico usamos la tecnología que nos ofrece \textbf{Twitter Bootstrap} \cite{cita3}, un framework \textbf{HTML} \cite{cita4}, \textbf{CSS} \cite{cita5} y \textbf{JavaScript} \cite{cita6} para lograr un diseño web adaptable (responsive design) entre los distintos dispositivos que puedan hacer uso de la aplicación web. \textbf{Ajax} \cite{cita7} y \textbf{jQuery} \cite{cita8} serán las herramientas elegidas para ofrecer una mejor interacción entre el usuario y la aplicación web. \\

La base de datos será gestionada con \textbf{MongoDB} \cite{cita9}, que es una base de datos orientada a documentos estilo JSON (incluye información del tipo de dato). \\

drawercloud hace uso de \textbf{EC2} \cite{cita_ec2} (Amazon Virtual Servers), que nos permite alojar la web gratuitamente durante un año y poder acceder a ella desde cualquier dispositivo que se conecte a la red. Como servidor web usamos \textbf{NGINX} \cite{cita_nginx} y \textbf{Gunicorn} \cite{cita_gunicorn} que es un WSGI de Python.\\

Este proyecto está bajo la licencia de software libre, lo que significa que los usuarios tienen la libertad de ejecutar, copiar, distribuir, estudiar, modificar y mejorar el software.

\clearpage


\thispagestyle{empty}


\begin{center}
{\large\bfseries drawercloud: aplicación web orientada al almacenamiento de archivos en la nube}\\
\end{center}
\begin{center}
José Manuel Rejón Santiago (student)\\
\end{center}

%\vspace{0.7cm}
\noindent{\textbf{Keywords}: Keyword1, Keyword2, Keyword3, ....}\\

\vspace{0.7cm}
\noindent{\textbf{Abstract}}\\

Write here the abstract in English.

\chapter*{}
\thispagestyle{empty}

\noindent\rule[-1ex]{\textwidth}{2pt}\\[4.5ex]

Yo, \textbf{José Manuel Rejón Santiago}, alumno de la titulación Grado en Ingeniería Informática de la \textbf{Escuela Técnica Superior
de Ingenierías Informática y de Telecomunicación de la Universidad de Granada}, con DNI 20077299R, autorizo la
ubicación de la siguiente copia de mi Trabajo Fin de Grado en la biblioteca del centro para que pueda ser
consultada por las personas que lo deseen.

\vspace{6cm}

\noindent Fdo: José Manuel Rejón Santiago

\vspace{2cm}

\begin{flushright}
Granada a 12 de Septiembre de 2017.
\end{flushright}


\chapter*{}
\thispagestyle{empty}

\noindent\rule[-1ex]{\textwidth}{2pt}\\[4.5ex]

D. \textbf{Prof. Dr. José María Guirao Miras (tutor)}, Profesor del Departamento de Lenguajes y Sistemas Informáticos de la Universidad de Granada.

\vspace{0.5cm}

\textbf{Informan:}

\vspace{0.5cm}

Que el presente trabajo, titulado \textit{\textbf{drawercloud, aplicación web orientada al almacenamiento de archivos en la nube}},
ha sido realizado bajo su supervisión por \textbf{José Manuel Rejón Santiago (alumno)}, y autoriza la defensa de dicho trabajo ante el tribunal
que corresponda.

\vspace{0.5cm}

Y para que conste, expiden y firman el presente informe en Granada a 12 de Septiembre de 2017.

\vspace{1cm}

\textbf{El tutor:}

\vspace{5cm}

\noindent \textbf{Prof. Dr. José María Guirao Miras}

\chapter*{Agradecimientos}
\thispagestyle{empty}

       \vspace{1cm}


En primer lugar quiero mostrar mi agradecimiento a mi tutor José María Guirao Miras por darme la oportunidad de realizar este proyecto y toda la ayuda y consejos recibidos. \\

También agradecer a mi familia y amigos/as que han estado apoyándome durante todo el desarrollo del proyecto como a lo largo de estos años de carrera. \\

Por último y no menos importante a mis compañeros durante esta experiencia, un gran apoyo siempre tanto dentro como fuera de clase.



