\documentclass[a4paper,11pt]{book}
%\documentclass[a4paper,twoside,11pt,titlepage]{book}
\usepackage{listings}
\usepackage[utf8]{inputenc}
\usepackage[spanish]{babel}

% \usepackage[style=list, number=none]{glossary} %
%\usepackage{titlesec}
%\usepackage{pailatino}

\decimalpoint
\usepackage{dcolumn}
\newcolumntype{.}{D{.}{\esperiod}{-1}}
\makeatletter
\addto\shorthandsspanish{\let\esperiod\es@period@code}
\makeatother


%\usepackage[chapter]{algorithm}
\RequirePackage{verbatim}
%\RequirePackage[Glenn]{fncychap}
\usepackage{fancyhdr}
\usepackage{graphicx}
\usepackage{afterpage}

\usepackage{longtable}

\usepackage[pdfborder={000}]{hyperref} %referencia

% ********************************************************************
% Re-usable information
% ********************************************************************
\newcommand{\myTitle}{Título del proyecto\xspace}
\newcommand{\myDegree}{Grado en ...\xspace}
\newcommand{\myName}{Nombre Apllido1 Apellido2 (alumno)\xspace}
\newcommand{\myProf}{Nombre Apllido1 Apellido2 (tutor1)\xspace}
\newcommand{\myOtherProf}{Nombre Apllido1 Apellido2 (tutor2)\xspace}
%\newcommand{\mySupervisor}{Put name here\xspace}
\newcommand{\myFaculty}{Escuela Técnica Superior de Ingenierías Informática y de
Telecomunicación\xspace}
\newcommand{\myFacultyShort}{E.T.S. de Ingenierías Informática y de
Telecomunicación\xspace}
\newcommand{\myDepartment}{Departamento de ...\xspace}
\newcommand{\myUni}{\protect{Universidad de Granada}\xspace}
\newcommand{\myLocation}{Granada\xspace}
\newcommand{\myTime}{\today\xspace}
\newcommand{\myVersion}{Version 0.1\xspace}


\hypersetup{
pdfauthor = {\myName (email (en) ugr (punto) es)},
pdftitle = {\myTitle},
pdfsubject = {},
pdfkeywords = {palabra_clave1, palabra_clave2, palabra_clave3, ...},
pdfcreator = {LaTeX con el paquete ....},
pdfproducer = {pdflatex}
}

%\hyphenation{}


%\usepackage{doxygen/doxygen}
%\usepackage{pdfpages}
\usepackage{url}
\usepackage{colortbl,longtable}
\usepackage[stable]{footmisc}
\usepackage{eurosym}
\usepackage{float}
\usepackage{listings}
\usepackage{color}
%\usepackage{index}

%\makeindex
%\usepackage[style=long, cols=2,border=plain,toc=true,number=none]{glossary}
% \makeglossary

% Definición de comandos que me son tiles:
%\renewcommand{\indexname}{Índice alfabético}
%\renewcommand{\glossaryname}{Glosario}

\pagestyle{fancy}
\fancyhf{}
\fancyhead[LO]{\leftmark}
\fancyhead[RE]{\rightmark}
\fancyhead[RO,LE]{\textbf{\thepage}}
\renewcommand{\chaptermark}[1]{\markboth{\textbf{#1}}{}}
\renewcommand{\sectionmark}[1]{\markright{\textbf{\thesection. #1}}}

\setlength{\headheight}{1.5\headheight}

\newcommand{\HRule}{\rule{\linewidth}{0.5mm}}
%Definimos los tipos teorema, ejemplo y definición podremos usar estos tipos
%simplemente poniendo \begin{teorema} \end{teorema} ...
\newtheorem{teorema}{Teorema}[chapter]
\newtheorem{ejemplo}{Ejemplo}[chapter]
\newtheorem{definicion}{Definición}[chapter]

\definecolor{gray97}{rgb}{1,1,1}
\definecolor{gray45}{rgb}{0.5,0.5,0.5}
\definecolor{gray}{rgb}{0.9,0.9,0.9}
\definecolor{dkgreen}{rgb}{0,0.6,0}
\definecolor{mauve}{rgb}{0.58,0,0.82}
\definecolor{verde}{rgb}{0.55,0.71,0.0}
\definecolor{azuloscuro}{rgb}{0.0,0.44,1.0}
\definecolor{naranja}{rgb}{0.91,0.45,0.32}
\definecolor{amarillo}{rgb}{0.94,0.86,0.51}
\definecolor{morado}{rgb}{0.0,0.2,0.4}
\definecolor{violeta}{rgb}{0.45,0.31,0.59}
\definecolor{verdelima}{rgb}{0.5,1.0,0.0}
\definecolor{rojo}{rgb}{0.89,0.26,0.2}
\definecolor{gris}{rgb}{0.5,0.5,0.5}
\definecolor{rosa}{rgb}{1.0,0.33,0.5}


\lstset{
	language=python,
	aboveskip=3mm,
	belowskip=3mm,
	showstringspaces=false,
	columns=flexible,
	basicstyle={\small\ttfamily},
	numbers=none,
	numberstyle=\tiny\color{gray},
	keywordstyle=\color{blue},
	commentstyle=\color{dkgreen},
	stringstyle=\color{mauve},
	breaklines=true,
	breakatwhitespace=false,
	tabsize=3
}

\lstset{ frame=Ltb,
     framerule=0.5pt,
     aboveskip=0.5cm,
     framextopmargin=3pt,
     framexbottommargin=3pt,
     framexleftmargin=0.1cm,
     framesep=0pt,
     rulesep=.4pt,
     backgroundcolor=\color{gray97},
     rulesepcolor=\color{black},
     %
     stringstyle=\ttfamily,
     showstringspaces = false,
     basicstyle=\scriptsize\ttfamily,
     commentstyle=\color{gray45},
     keywordstyle=\bfseries,
     %
     numbers=left,
     numbersep=6pt,
     numberstyle=\tiny,
     numberfirstline = false,
     breaklines=true,
   }
 
% minimizar fragmentado de listados
\lstnewenvironment{listing}[1][]
   {\lstset{#1}\pagebreak[0]}{\pagebreak[0]}

\lstdefinestyle{CodigoC}
   {
	basicstyle=\scriptsize,
	frame=single,
	language=C,
	numbers=left
   }
\lstdefinestyle{CodigoC++}
   {
	basicstyle=\small,
	frame=single,
	backgroundcolor=\color{gray30},
	language=C++,
	numbers=left
   }
 
\lstdefinestyle{Consola}
   {basicstyle=\scriptsize\bf\ttfamily,
    backgroundcolor=\color{gray30},
    frame=single,
    numbers=none
   }


\newcommand{\bigrule}{\titlerule[0.5mm]}


%Para conseguir que en las páginas en blanco no ponga cabecerass
\makeatletter
\def\clearpage{%
  \ifvmode
    \ifnum \@dbltopnum =\m@ne
      \ifdim \pagetotal <\topskip
        \hbox{}
      \fi
    \fi
  \fi
  \newpage
  \thispagestyle{empty}
  \write\m@ne{}
  \vbox{}
  \penalty -\@Mi
}
\makeatother

\usepackage{pdfpages}
\begin{document}
\begin{titlepage}
 
 
\newlength{\centeroffset}
\setlength{\centeroffset}{-0.5\oddsidemargin}
\addtolength{\centeroffset}{0.5\evensidemargin}
\thispagestyle{empty}

\noindent\hspace*{\centeroffset}\begin{minipage}{\textwidth}

\centering
\includegraphics[width=0.9\textwidth]{imagenes/logo_ugr.jpg}\\[1.4cm]

\textsc{ \Large TRABAJO FIN DE GRADO\\[0.2cm]}
\textsc{ Grado en Ingeniería Informática}\\[1cm]
% Upper part of the page
% 
% Title
{\Huge\bfseries drawercloud\\
}
\noindent\rule[-1ex]{\textwidth}{3pt}\\[3.5ex]
{\large\bfseries Aplicación web orientada al almacenamiento de archivos en la nube}
\end{minipage}

\vspace{2.5cm}
\noindent\hspace*{\centeroffset}\begin{minipage}{\textwidth}
\centering

\textbf{Autor}\\ {José Manuel Rejón Santiago (alumno)}\\[2.5ex]
\textbf{Director}\\
{Prof. Dr. José María Guirao Miras (tutor)}\\[2cm]
\includegraphics[width=0.3\textwidth]{imagenes/etsiit_logo.png}\\[0.1cm]
\textsc{Escuela Técnica Superior de Ingenierías Informática y de Telecomunicación}\\
\textsc{---}\\
Granada, Septiembre de 2017
\end{minipage}
%\addtolength{\textwidth}{\centeroffset}
%\vspace{\stretch{2}}
\end{titlepage}



\chapter*{}
%\thispagestyle{empty}
%\cleardoublepage

%\thispagestyle{empty}

\input{portada/portada_2}

\clearpage
\thispagestyle{empty}

\begin{center}
{\large\bfseries drawercloud: aplicación web orientada al almacenamiento de archivos en la nube}\\
\end{center}
\begin{center}
José Manuel Rejón Santiago (alumno)\\
\end{center}

%\vspace{0.7cm}
\noindent{\textbf{Palabras clave}: aplicación web, cloud, pc, tablet, smartphone, servidor web}\\

\vspace{0.5cm}
\noindent{\textbf{Resumen}}\\

\textbf{drawercloud} es una aplicación web para el alojamiento de archivos en la nube (cloud). Los usuarios podrán realizar diversas tareas sobre sus archivos, como por ejemplo: Añadir/Eliminar archivos en la nube, reproducir archivos multimedia, organizar archivos en directorios, compartir archivos, descargar en un dispositivo (pc, tablet, smartphone)... De este modo, conseguiremos tener nuestros archivos almacenados en un servidor web, ahorrando espacio en nuestros dispositivos y con la seguridad de tener nuestro material a salvo en otro lugar.\\

\clearpage


\thispagestyle{empty}


\begin{center}
{\large\bfseries drawercloud: aplicación web orientada al almacenamiento de archivos en la nube}\\
\end{center}
\begin{center}
José Manuel Rejón Santiago (student)\\
\end{center}

%\vspace{0.7cm}
\noindent{\textbf{Keywords}: web application, cloud, pc, tablet, smartphone, web server}\\

\vspace{0.7cm}
\noindent{\textbf{Abstract}}\\

drawecloud is a web application for file hosting in the cloud. The users can work with their files, for example: add/delete files in the cloud, play multimedia files, organize the files in different directories, share files, download to devices (pc, tablet, smartphone)... With this, we'll achieve the task of having all our files stored in a web server, saving space in our devices and with the security of having eveything safe in another place.

\chapter*{}
\thispagestyle{empty}

\noindent\rule[-1ex]{\textwidth}{2pt}\\[4.5ex]

Yo, \textbf{José Manuel Rejón Santiago}, alumno de la titulación Grado en Ingeniería Informática de la \textbf{Escuela Técnica Superior
de Ingenierías Informática y de Telecomunicación de la Universidad de Granada}, con DNI 20077299R, autorizo la
ubicación de la siguiente copia de mi Trabajo Fin de Grado en la biblioteca del centro para que pueda ser
consultada por las personas que lo deseen.

\vspace{6cm}

\noindent Fdo: José Manuel Rejón Santiago

\vspace{2cm}

\begin{flushright}
Granada a 12 de Septiembre de 2017.
\end{flushright}


\chapter*{}
\thispagestyle{empty}

\noindent\rule[-1ex]{\textwidth}{2pt}\\[4.5ex]

D. \textbf{Prof. Dr. José María Guirao Miras (tutor)}, Profesor del Departamento de Lenguajes y Sistemas Informáticos de la Universidad de Granada.

\vspace{0.5cm}

\textbf{Informan:}

\vspace{0.5cm}

Que el presente trabajo, titulado \textit{\textbf{drawercloud, aplicación web orientada al almacenamiento de archivos en la nube}},
ha sido realizado bajo su supervisión por \textbf{José Manuel Rejón Santiago (alumno)}, y autoriza la defensa de dicho trabajo ante el tribunal
que corresponda.

\vspace{0.5cm}

Y para que conste, expiden y firman el presente informe en Granada a 12 de Septiembre de 2017.

\vspace{1cm}

\textbf{El tutor:}

\vspace{5cm}

\noindent \textbf{Prof. Dr. José María Guirao Miras}

\chapter*{Agradecimientos}
\thispagestyle{empty}

       \vspace{1cm}


En primer lugar quiero mostrar mi agradecimiento a mi tutor José María Guirao Miras por darme la oportunidad de realizar este proyecto y toda la ayuda y consejos recibidos. \\

También agradecer a mi familia y amigos/as que han estado apoyándome durante todo el desarrollo del proyecto como a lo largo de estos años de carrera. \\

Por último y no menos importante a mis compañeros durante esta experiencia, un gran apoyo siempre tanto dentro como fuera de clase.




%\frontmatter
\tableofcontents
\listoffigures
\listoftables
%
%\mainmatter
%\setlength{\parskip}{5pt}

\chapter{Introducción}

El objetivo de este proyecto es el desarrollo de una aplicación web que proporcione al usuario la posibilidad de almacenar sus archivos en la nube. La aplicación web cuenta con opciones como el almacenamiento en un lugar seguro, la organización por carpetas, la posibilidad de crear grupos de trabajo donde compartir los archivos con compañeros, entre otras más.

\section{Motivación}

Unos de los problemas que surgen en el uso de nuestros dispositivos es el almacenamiento. Es cierto que el precio del \textbf{gigabyte} es cada vez más barato e incluso más rápido (como por ejemplo en los discos duros SSD, que están dotados de un gran incremento de velocidad, además de consumir menos energía) pero, aunque en un ordenador el almacenamiento puede no ser un problema, si que lo es si hablamos de dispositivos móviles como smartphones o tablets. A día de hoy, encontramos dispositivos que no permiten la insercción de tarjetas de almacenamiento flash, lo cual nos obliga a conformarnos con el espacio que el fabricante pone a nuestra disposición (o a comprar el modelo con más capacidad, pero obviamente más caro). \textbf{drawercloud} viene a poner solución a este problema, ofreciendo un espacio personal en un servidor web dedicado para dicha aplicación, de modo que nuestros datos permanecen en tal servidor sin ocupar memoria en nuestros dispositivos. \\

¿Significa esto que solo es una solución para dispositivos que cuentan con un espacio muy limitado? No, ya que otro problema que puede surgirnos con mucha facilidad es la pérdida de información por diversas causas (disco duro roto, pérdida del dispositivo, virus, etc) y para ello este proyecto también ofrece una solución a dicho problema. Podemos almacenar nuestros archivos tanto en el disco duro propio del dispositivo en cuestión, como en el espacio ofrecido en el servidor web, de modo que si nos vemos afectados por alguno de los motivos mencionados, o bien por otros que afectan a la desaparición de nuestra información, tengamos una copia en otro lugar y no perdamos nuestros archivos. \\

Otro caso en el que esta aplicación resulta interesante es que podemos disponer de nuestros datos en cualquier parte. A modo de ejemplo, podemos pensar en un archivo que estamos modificando en casa y en nuestro puesto de trabajo. Almacenando dicho archivo en la nube nos tenemos que olvidar de mover la información a través de, por ejemplo, un pen drive o de tener una copia en casa y otra en la oficina y andar actualizando ambas cada vez que queramos trabajar. Simplemente entraremos con nuestra cuenta a \textbf{drawercloud} desde cualquier dispositivo y tendremos acceso al archivo en cuestión. Además, proporciona un método para compartir archivos y crear grupos de trabajo, de modo que si sobre ese archivo están trabajando, por ejemplo, 3 personas más, podrán ver las modificaciones que hemos realizado directamente sobre el archivo alojado en el servidor web.

\input{capitulos/02_Analisis}
\chapter{Diseño}

En este apartado se comentarán los aspectos más importantes para el desarrollo de la aplicación. Para estudiar este apartado vamos a dividir el contenido en dos grupos: servidor (backend) y cliente (frontend).

\section{Backend}
El \textbf{backend} se encarga de recibir los datos desde el frontend y procesar dichos datos. \\

\subsection{Patrón Modelo-Vista-Controlador (MVC)}
Para el desarrollo de la aplicación se plantea el uso del patrón de arquitectura de software MVC, el cual se basa en separar los datos de la lógica de la aplicación y su interfaz. Ésto facilita el desarrollo y el posterior mantenimiento de la aplicación.
\begin{itemize}
	\item \textbf{Modelo} (Model): \textbf{información} con la que trabaja el sistema. Se encarga de gestionar los accesos a dicha información, tanto consultas como actualizaciones. Se corresponde con la base de datos.
	\item \textbf{Vista} (View): se refiere a los datos que se van a mostrar y como mostrarlos. La vista será el \textbf{frontend} de nuestra aplicación.
	\item \textbf{Controlador} (Controller): responde a \textbf{eventos} (generalmente los proporcionará el usuario) y llama al modelo cuando se realiza alguna solicitud sobre la información. También podrá enviar solicitudes a la vista si se solicita un cambio en la forma que mostrar la información. \\
\end{itemize}

\begin{figure}[H]
	\centering
	\includegraphics[width=0.7\textwidth]{imagenes/img_mvc}
	\caption{Diagrama Model-View-Controller.}
	\label{fig:img_mvc}
\end{figure}

\subsection{Django}
La decisión de usar el framework \textbf{Django} viene dada por su uso de la arquitectura MVC y su disponibilidad para trabajar con Python. Debido a que el controlador es manejado por el mismo framework y la parte más importante se produce en los modelos, las plantillas y las vistas, Django es conocido como un Framework MTV (Model- View - Template) \cite{cita_mvt}.
\begin{itemize}
	\item \textbf{Modelo} (Model): \textbf{información} con la que trabaja el sistema. Se encarga de gestionar los accesos a dicha información, tanto consultas como actualizaciones. Se corresponde con la base de datos.
	\item \textbf{Vista} (View): esta capa contiene la \textbf{lógica} que accede al modelo y la delega a la plantilla apropiada: actúa como puente entre el modelos y las plantillas.
	\item \textbf{Plantilla} (Template): se relaciona con la \textbf{presentación} de la información: como algunas cosas son mostradas sobre una página web u otro tipo de documento. \\
\end{itemize}

\begin{figure}[H]
	\centering
	\includegraphics[width=0.7\textwidth]{imagenes/img_mvt}
	\caption{Diagrama Model-View-Template.}
	\label{fig:img_mvt}
\end{figure}

Django nos permite crear sitios web complejos ofreciendo diversos componentes que podemos usar para el desarrollo de nuestra aplicación.

\subsection{Sistema gestor de base de datos}
El sistema gestor de base de datos es el encargado de proporcionar la capa de acceso a la información de la base de datos. Haciendo uso del ORM de Django, utilizaremos un conector para la base de datos \textbf{MongoDB}.

\section{Frontend}
El \textbf{frontend} se refiere a la parte del sistema que se encarga de interactuar con los usuarios y mandar las peticiones al backend. Para interactuar con los usuarios hacemos uso de tecnologías web como \textbf{HTML}, \textbf{CSS} y \textbf{Javascript}. Este sistema trabaja todo el tiempo recibiendo peticiones de usuarios, por lo tanto usando las tecnologías mencionadas y añadiendo otras, como \textbf{jQuery}, haremos de nuestra interfaz un entorno intuitivo para los usuarios. \\

\subsection{jQuery}
\textbf{jQuery} es una librería de Javascript la cual ofrece la posibilidad de manipular los documentos HTML y manejar eventos, dotando a la web de una mayor dinamicidad. Es compatible con la mayoría de los navegadores actuales, tanto en pc como en dispositivos móviles. \\

\subsection{AJAX}
Por otro lado \textbf{AJAX} nos brinda la oportunidad de cargar datos desde el backend sin la necesidad de recargar toda la página del navegador. \\

\subsection{Twitter Bootstrap}
Otra tecnología usada de cara a la interacción con los usuarios de nuestra aplicación es la que nos aporta \textbf{Twitter Bootstrap}. Twitter Bootstrap es un framework CSS que nos facilita la tarea del diseño de la aplicación. Con esta tecnología podremos crear diseños muy elegantes y limpios, además de dotar a la aplicación web de un diseño adaptativo (responsive design) y así poder disfrutar de nuestra aplicación sin problemas en cualquier dispositivo y tamaño de pantalla. \\

\chapter{Implementación}

Tras haber visto la parte del diseño toca trabajar la implementación. En este apartado profundizaremos sobre el entorno de trabajo usado y la estructura de la aplicación.

\section{Entorno de trabajo}
Django nos da la posibilidad de trabajar con un "entorno virtual". Estos entornos son creados con el objetivo de aislar recursos, como son las librerías y el entorno de ejecución, del sistema principal o de otros entornos virtuales. Esto quiere decir que podemos tinstalar distintas versiones de una librería sin que se provoquen conflictos. \\
Para usar el entorno de trabajo hacemos uso de la herramienta \textbf{virtualenv}. Los comandos para crear un entorno y activarlo son los siguientes:

\begin{itemize}
	\item \textbf{Instalar virtualenv:} sudo pip install virtualenv
	\item \textbf{Crear entorno de trabajo:} virtualenv drawercloud\_env
	\item \textbf{Activar el entorno de trabajo:} source drawercloud\_env/bin/activate
\end{itemize}

Como podemos ver, para la instalación de virtualenv se ha usado \textbf{pip}. pip es la herramienta recomendada para instalar y administrar los paquetes de python. Los comandos que se usarán con más frecuencia son los que proceden:

\begin{itemize}
	\item \textbf{Instalar un paquete:} pip install <nombre\_paquete>
	\item \textbf{Desinstalar un paquete:} pip uninstall <nombre\_paquete>
	\item \textbf{Actualizar un paquete:} pip install <nombre\_paquete> --upgrade
	\item \textbf{Listar los paquetes instalados:} pip freeze
\end{itemize}

\textbf{Importante:} Todos los comandos se deberán usar con el entorno de trabajo activado.

\section{Estructura del proyecto}
A continuación se mostrarán las partes en las que se compone la estructura del proyecto. \\

\subsection{Fichero de configuración settings.py}
El fichero \textbf{settings.py} es el encargado de controlar la configuración del proyecto en Django. A continuación mencionaremos las opciones más destacables de este fichero según este proyecto:

\begin{itemize}
	\item \textbf{DEBUG:} cuando la variable DEBUG se encuentre igualada a \textbf{True} querrá decir que el sistema se ejecuta en modo "depuración", el cual nos proporcia los detalles de un error cuando éste se produzca. Cuando la aplicación se ejecute en "producción", a la variable le asignaremos el valor \textbf{False} por motivos de seguridad.
	
	\item \textbf{SECRET\_KEY:} es la llave secreta de nuestro proyecto y sirve para encriptar la información dentro de la base de datos.
	
	\item \textbf{INSTALLED\_APPS:} es una lista en la que se indica las aplicaciones habilitadas para este proyecto. Por ejemplo, para usar el administrador de Django añadimos la sentencia \textbf{'django.contrib.admin'}.
	
	\item \textbf{MIDDLEWARE:} es un framework que se usa como enlace en el proceso de solicitud/respuesta.
	
	\item \textbf{STATICFILES\_DIRS}: se indica la ruta a la carpeta \textbf{static} de nuestro proyecto. Dicha carpeta es la que contiene archivos relacionados con el frontend, como pueden ser los archivos CSS o los archivos Javascript.
	
	\item \textbf{connect(database\_name)}: esta función se ocupa de conectar con la base de datos indicada en el parámetro database\_name. \textbf{connect} es una función de MongoDB, para usarla haremos las siguientes importaciones:
	\begin{itemize}
		\item from pymongo import MongoClient
		\item from mongoengine import connect				
	\end{itemize}
	
	\item \textbf{REGISTRATION\_OPEN}: esta variable se igualará a \textbf{True} para permitir el registro de usuarios.
	
	\item \textbf{LOGIN\_REDIRECT\_URL}: indicamos la ruta a la que debe redirigirnos la aplicación una vez se ha realizado el log in con éxito.
	
	\item \textbf{LOGIN\_URL}: con esta variable decimos al sistema donde debemos aparecer en caso de no haber realizado el log in, o bien si se intenta acceder a páginas que requieren haberse logueado previamente.
\end{itemize}

Para obtener más información acerca del fichero de configuración settings.py de Django podemos visitar el enlace de la cita \cite{cita_django_settings}.

\subsection{Fichero de configuración urls.py}
En este fichero se indican las \textbf{URLs} que se van a usar en la aplicación. Tendremos dos ficheros urls.py. El que ahora comentamos se encontrará en la misma carpeta que el fichero settings.py mientras que el otro se encontrará en la carpeta del proyecto, la cual contiene los archivos como views.py, la carpeta static o la carpeta templates. En este apartado nos referimos al archivo urls.py almacenado junto a settings.py. En éste indicaremos las rutas a la página de administración, la ruta al resto de urls del proyecto y la ruta necesaria para el registro de usarios mediante django-registration-redux: \\

\begin{lstlisting}[language=python]
	urlpatterns = [
		url(r'^admin/', admin.site.urls),
		url(r'^proyecto/', include('proyecto.urls')),
		url(r'^accounts/', include('registration.backends.simple.urls')),
	]
\end{lstlisting}

\subsection{Fichero views.py}
El archivo \textbf{views.py} contiene las funciones que recibirán peticiones web y responderán a dichas peticiones también con un resultado web. Las respuestas prodrán ser un código HTML, una imagen, reproducir una canción, etc. El propio archivo contiene cualquier lógica que sea necesaria para poder devolver una respuesta. A modo de ejemplo vemos una vista simple como es la página principal de la sección "Multimedia" de nuestra aplicación. \\

\begin{lstlisting}[language=python]
	@login_required(login_url='/accounts/login/') #Requiere estar logueado para usar la funcion
	def multimedia(request):
		return render(request, 'multimedia.html', {'pagina_actual':'Multimedia'}) #Renderiza el fichero HTML multimedia.html
\end{lstlisting}

\subsection{Fichero models.py}
Recordamos que estamos usando una base de datos MongoDB. MongoDB es una base de datos orientada a documentos. Esto quiere decir que los datos no se guardan en registros, sino que se guardan en los denominados documentos. Éstos se almacenan en BSON, que nos es más que una representación binaria de JSON. \\

El fichero encargado para la gestión de documentos es el fichero \textbf{models.py}. En dicho fichero detallamos la estructura de los documentos que vayamos a crear. Por ejemplo, para el documento \textbf{Archivo} (usado para almacenar los archivos que se suban a drawercloud) tenemos la siguiente disposición: \\

\begin{lstlisting}[language=python]
	class Archivo(Document):
		id_archivo = DecimalField()
		nombre = StringField()
		tipo_archivo = StringField()
		archivo = FileField()
		fecha_subida = StringField()
		propietario = StringField()
		tam_archivo = DecimalField()
		favorito = BooleanField()
\end{lstlisting}


\subsection{Fichero forms.py}
En el archivo \textbf{forms.py} escribiremos toda la funcionalidad necesaria para gestionar los documentos de la base de datos. Las acciones tales como crear un usuario, almacenar un archivo, añadir un usuario a un grupo de trabajo... vendrán descritas en este fichero. Por ejemplo, la función para guardar un nuevo usuario quedaría de la siguiente manera: \\

\begin{lstlisting}[language=python]
	def save(self, _username):
		u = Usuario()
		
		u.username = _username
		u.id_username = Usuario.objects.count() + 1
		u.img_perfil = -1
		
		u.save()
		
		return u
\end{lstlisting}

\subsection{Fichero urls.py}
Nos encontramos aquí con el otro fichero \textbf{urls.py}. Dicho fichero contiene todas las urls de todas las funciones que se encuentran en el fichero views.py. Por ejemplo, para la función que se ocupa de descargar un archivo tendremos algo tal que así: \\

\begin{lstlisting}[language=python]
	url(r'^descargarArchivo/$', views.descargarArchivo, name='descargarArchivo'),
\end{lstlisting}

Analizando la sentencia, tenemos que el primer argumento es la url que tendrá en la web. El segundo argumento indica el nombre de la función en el fichero views.py. Por último, el tercer argumento indica el nombre con el cual nos podremos referir a la función cuando queramos invocarla, por ejemplo, desde un fichero HTML o Javascript. \\

\subsection{Directorio templates}
Este directorio contiene todos los archivos \textbf{HTML} que forman la vista de la aplicación. \\ 

\subsection{Directorio static}
Este directorio contiene todos los archivos \textbf{CSS, Javascript, imágenes, ...} que complementan a los archivos HTML almacenados en el directorio templates y que juntos forman el frontend. \\ 
\chapter{Pruebas}

Se han realizado distintas pruebas para estudiar el comportamiendo de la aplicación web (dichas pruebas se realizaron con la última versión de la aplicación). Para realizar las pruebas se ha usado la ayuda de un grupo de 10 personas, las cuales han estado interactuando con la aplicación durante unos 10 minutos aproximadamente. Los resultados han sido los siguientes:

\section{Resultados de las pruebas con usuarios}
La prueba se iniciaba realizando dos preguntas al usuario:

\begin{itemize}
	\item ¿Sabes lo que es la nube?
	\item ¿Has usado alguna aplicación que use esta tecnología?
\end{itemize}

El 100\% de los usuarios conocía que era la nube y solo tres de ellos no habían usado nunca este tipo de aplicación. \\


Una vez se realizaban las preguntas se dejaba al usuario usar la aplicación web solo, de modo que fuera él mismo quien resolviera sus dudas. Durante la prueba se hizo uso de todas las acciones principales, como subir archivos, organizarlos en directorios, consultar el apartado multimedia, compartir archivos, agregar a favoritos y trabajar con los grupos de trabajo. \\

\subsubsection{¿Qué podemos destacar de las pruebas?}

\begin{itemize}{}
	\item Se puede determinar que se usa una interfaz intuitiva ya que los usuarios se desenvolvieron rápida y correctamente visitando cada página y haciendo uso de las funciones básicas. En el caso de los usuarios que no habían usado nunca una aplicación de este tipo no hubo problemas destacables tampoco.
	\item El uso del botón derecho por encima del botón "Más" para mostrar las opciones de directorios/archivos. Apenas se llegó a usar dicho botón.
	\item El botón con la imagen de perfil de usuario que muestra las opciones de Usuario, Ayuda y Salir pasó desapercibido para un total de 7 usuarios.
	\item 4 usuarios intentaron seleccionar archivos, pero esta funcionalidad no está disponible.
	\item La acción de arrastrar para mover un archivo a una carpeta la intentaron 9 personas, pero esta opción tampoco se desarrolló en esta versión del proyecto.
	\item 3 usuarios intentaron subir varios archivos a la vez y al igual que antes, esta versión no cuenta con tal opción.
\end{itemize}

En resumen podemos concluir que la aplicación tiene una interfaz intuitiva y que cubre las necesidades básicas para desempeñar su función de almacenador archivos en la nube. Se deberían corregir sin embargo los problemas más repetidos.
\chapter{Conclusiones y trabajos futuros}

\section{Conclusiones}
Tras la finalización del proyecto se ha obtenido un prototipo funcional de la aplicación web, que cumple con casi todas las expectativas y los objetivos marcados desde el principio, además de una completa memoria del proyecto. \\
 
Gracias a las pruebas realizadas con usuarios podemos determinar que...... \\

En definitiva tenemos que \textbf{drawercloud} cumple con los requisitos para efectuar su función principal de alojar archivos en la nube, a excepción del espacio disponible de almacenamiento actual que es insuficiente para una aplicación de estas características. Esto se debe a que el espacio que nos ofrece Amazon es solamente de 2GB, debiéndose esto a que se trata de la versión gratuita. No obstante esto tendría solución comprando más espacio de almacenamiento y ofreciendo a los clientes distintas versiones, las cuales serían de pago para grandes cantidades de espacio y una gratuita que ofrecería menos capacidad. \\

\subsection{Conocimientos adquiridos}
Gracias al análisis y al desarrollo de este proyecto se han adquirido conocimientos en diversos ámbitos del desarrollo de software, en su mayoría en el sector web. Entre los conocimientos adquiridos podemos nombrar: 

\begin{itemize}
	\item \textbf{Python:}
	\item \textbf{Django:}
	\item \textbf{Arquitectura MVC:}
	\item \textbf{Tecnologías web:}
	\item \textbf{Pruebas con usuarios:}
	\item \textbf{Desarrollo de documentación:}
\end{itemize}

\section{Trabajos futuros}
\chapter{Manual de usuario}

\section{Acceso a la aplicación}
La primera página que se carga de nuestra aplicación web es la que se refiere al login. En la imagen \ref{fig:login} podemos consultar el aspecto de dicha página. Observamos dos entradas de texto, una para insertar el nombre de usuario y otra para la contraseña. \\

Para usuarios que no estén registrados se presenta la opción de \textbf{Crear una nueva cuenta}, en la cual debemos rellenar los datos requeridos: nombre de usuario, dirección de correo electrónico, contraseña y confirmar contraseña. En la imagem \ref{fig:registro}

\begin{figure}[H]
	\centering
	\includegraphics[width=1\textwidth]{imagenes/login}
	\caption{Acceso a la aplicación. Página de login}
	\label{fig:login}
\end{figure}

\begin{figure}[H]
	\centering
	\includegraphics[width=1\textwidth]{imagenes/registro}
	\caption{Acceso a la aplicación. Página de registro}
	\label{fig:registro}
\end{figure}

\section{Página principal - Documentos}
La página principal es la sección \textbf{Documentos} \ref{fig:documentos}. En esta página nos encontraremos con una estructura de directorios y archivos. La vista en que se muestran los directorios y archivos se podrá cambiar haciendo uso de los botones de la imagen \ref{fig:bts_cambiar_vista}, estando disponibles dos tipos de vista: lista \ref{fig:documentos} o iconos \ref{fig:documentos_iconos}.


\begin{figure}[H]
	\centering
	\begin{subfigure}{0.4\textwidth}
		\centering
		\includegraphics[width=.4\linewidth]{imagenes/bt_cambiar_vista_iconos}
		\caption{Cambiar a la vista \textbf{iconos}}
		\label{fig:bt_cambiar_vista_iconos}
	\end{subfigure}%
	\begin{subfigure}{0.4\textwidth}
		\centering
		\includegraphics[width=.4\linewidth]{imagenes/bt_cambiar_vista_lista}
		\caption{Cambiar a la vista \textbf{lista}}
		\label{fig:bt_cambiar_vista_lista}
	\end{subfigure}
	\caption{Botones disponibles para cambiar la vista}
	\label{fig:bts_cambiar_vista}
\end{figure}

\begin{figure}[H]
	\centering
	\includegraphics[width=1\textwidth]{imagenes/documentos}
	\caption{Documentos. Aspecto de la página principal}
	\label{fig:documentos}
\end{figure}

\begin{figure}[H]
	\centering
	\includegraphics[width=1\textwidth]{imagenes/documentos_iconos}
	\caption{Documentos. Aspecto de la página principal}
	\label{fig:documentos_iconos}
\end{figure}

En \textbf{Documentos} vamos a poder almacenar los archivos que deseemos, podiendo organizar tales archivos en directorios. Para \textbf{crear directorios} o \textbf{subir archivos} hacemos uso de las opciones que aparecen a la derecha de la página documentos \ref{fig:opciones_documentos}.

\begin{figure}[H]
	\centering
	\includegraphics[width=0.4\textwidth]{imagenes/opciones_documentos}
	\caption{Documentos. Opciones de la página Documentos}
	\label{fig:opciones_documentos}
\end{figure}

Los archivos y directorios disponen de \textbf{funcionalidades} propias, como la posibilidad de descargar un archivo, copiar un directorio, etc. Para acceder a estas opciones podemos elegir entre hacer \textbf{click derecho} sobre los directorios/archivos o seleccionar el botón \textbf{Más} disponible para la vista lista.

\subsection{Opciones para archivos}
\begin{itemize}
	\item \textbf{Descargar:} permite transferir el archivo desde la nube hasta el dispositivo usado.
	\item \textbf{Compartir:} dar permisos de lectura sobre el archivo a otro usuario.
	\item \textbf{Cambiar nombre:} renombrar un archivo.
	\item \textbf{Copiar en:} realiza una copia de un archivo.
	\item \textbf{Mover a:} mueve el archivo a otro directorio.
	\item \textbf{Añadir a favoritos:} destaca el archivo sobre el resto.
	\item \textbf{Eliminar:} borra de la nube todos los datos relacionados con el archivo.
\end{itemize} 

\subsection{Opciones para directorios}
\begin{itemize}
	\item \textbf{Cambiar nombre:} renombrar un directorio.
	\item \textbf{Copiar en:} realiza una copia de un directorio.
	\item \textbf{Mover a:} mueve el directorio a otro directorio.
	\item \textbf{Eliminar:} borra de la nube el directorio y todo su contenido.
\end{itemize}

\section{Multimedia}
El apartado \textbf{Multimedia} consta de 3 grupos distintos de archivos multimedia: \textbf{Música, Imágenes y Vídeos} \ref{fig:multimedia}. El contenido de estas carpetas se actualiza de forma automática con la subida de un nuevo fichero. Seleccionando cualquiera de los grupos accederemos a su tipo de contenido.

\begin{figure}[H]
	\centering
	\includegraphics[width=1\textwidth]{imagenes/multimedia}
	\caption{Multimedia. Selección del tipo de contenido.}
	\label{fig:multimedia}
\end{figure}

Dentro de cada grupo obtendremos una lista de los archivos correspondientes. Por ejemplo, si seleccionamos la carpeta imágenes nos encontraremos con todos los archivos de dicho tipo. En esta carpeta volveremos a encontrarnos con las opciones de crear carpeta y subir archivos \ref{fig:opciones_documentos} y las mismas acciones sobre archivos y directorios (copiar, descargar, etc).

\section{Compartir}
\textbf{Compartir} es la página que se encarga de mostrar los archivos que están siendo compartidos. Para mostrar tales archivos se ha dividido la página en dos partes: \textbf{Archivos compartidos por mi} \ref{fig:compartido_por_mi} y \textbf{Archivos compartidos conmigo} \ref{fig:compartido_conmigo}.

\begin{figure}[H]
	\centering
	\includegraphics[width=1\textwidth]{imagenes/compartido_por_mi}
	\caption{Compartir. Archivos compartidos por mi.}
	\label{fig:compartido_por_mi}
\end{figure}

\begin{figure}[H]
	\centering
	\includegraphics[width=1\textwidth]{imagenes/compartido_conmigo}
	\caption{Compartir. Archivos compartidos conmigo.}
	\label{fig:compartido_conmigo}
\end{figure}

Los archivos realizarán acciones distintas según el apartado en el que estemos. Vemos a continuación que se puede realizar según si es compartido por mi o compartido conmigo.

\subsection{Opciones para archivos compartidos por mi}
\begin{itemize}
	\item \textbf{Descargar:} permite transferir el archivo desde la nube hasta el dispositivo usado.
	\item \textbf{Dejar de compartir:} retira los permisos de lectura sobre el archivo a otro usuario.
	\item \textbf{Cambiar nombre:} renombrar un archivo.
	\item \textbf{Añadir a favoritos:} destaca el archivo sobre el resto.
\end{itemize} 

\subsection{Opciones para archivos compartidos conmigo}
\begin{itemize}
	\item \textbf{Descargar:} permite transferir el archivo desde la nube hasta el dispositivo usado.
	\item \textbf{Dejar de compartir:} nos retira los permisos de lectura sobre el archivo.
	\item \textbf{Añadir a mi nube:} nos permite realizar una copia del archivo en nuestra propia cuenta y nos convierte en propietario de ella.
\end{itemize}

\section{Favoritos}
En la página \textbf{Favoritos} \ref{fig:favoritos} tenemos los archivos que, como el propio nombre de la página indica, han sido marcados como favoritos. Esto nos permite acceder de forma rápida al contenido que hayamos dotado de esta propiedad.

\begin{figure}[H]
	\centering
	\includegraphics[width=1\textwidth]{imagenes/favoritos}
	\caption{Favoritos. Archivos marcados como favoritos.}
	\label{fig:favoritos}
\end{figure}

Las acciones que los archivos podrán realizar desde este apartado son las siguientes: 

\subsection{Opciones para archivos favoritos}
\begin{itemize}
	\item \textbf{Descargar:} permite transferir el archivo desde la nube hasta el dispositivo usado.
	\item \textbf{Compartir:} dar permisos de lectura sobre el archivo a otro usuario.
	\item \textbf{Cambiar nombre:} renombrar un archivo.
	\item \textbf{Eliminar de favoritos:} elimina la condición de favorito del archivo.
\end{itemize}

\section{Grupo de trabajo}
Por último tenemos la página \textbf{Grupo de trabajo}. Esta sección ha sido creada con el objetivo de crear espacios comunes para distintos usuarios, de modo que todos los participantes del grupo tienen los mismos permisos sobre éste. La primera página que nos aparece al acceder a los grupos de trabajo \ref{fig:grupo_trabajo_principal} nos da a elegir entre \textbf{Ver mis grupos} o \textbf{Crear un grupo}.

\begin{figure}[H]
	\centering
	\includegraphics[width=1\textwidth]{imagenes/grupo_trabajo_principal}
	\caption{Grupo de trabajo. Ver mis grupos o Crear un grupo.}
	\label{fig:grupo_trabajo_principal}
\end{figure}

\subsection{Crear un grupo}
Si hemos elegido la opción \textbf{Crear un grupo}, nos aparecerá una ventana como la que vemos en la imagen \ref{fig:crear_grupo}. En la ventana emergente indicaremos el nombre del grupo y pulsaremos en \textbf{Crear grupo} para finalizar.

\begin{figure}[H]
	\centering
	\includegraphics[width=1\textwidth]{imagenes/crear_grupo}
	\caption{Grupo de trabajo. Crear un grupo.}
	\label{fig:crear_grupo}
\end{figure}

\subsection{Ver mis grupos}
Si por el contrario hemos elegido la opción \textbf{Ver mis grupos}, nos aparecerá una ventana como la que vemos en la imagen \ref{fig:ver_mis_grupos}. En la ventana emergente nos aparece una lista con los grupos a los que pertenecemos. Accederemos al interior de cada grupo haciendo click sobre el elegido.

\begin{figure}[H]
	\centering
	\includegraphics[width=1\textwidth]{imagenes/ver_mis_grupos}
	\caption{Grupo de trabajo. Ver mis grupos.}
	\label{fig:ver_mis_grupos}
\end{figure}


\subsection{Contenido de un grupo de trabajo}
Una vez hemos accedido a un grupo de trabajo veremos algo similar a lo que nos muestra la figura \ref{fig:grupo_trabajo}. Contamos de nuevo con la estructura repetida en las otras páginas: una arquitectura de archivos y directorios (mostrados en lista o iconos) y las posibles acciones en un menú a la derecha.

\begin{figure}[H]
	\centering
	\includegraphics[width=1\textwidth]{imagenes/grupo_trabajo}
	\caption{Grupo de trabajo. Contenido de un grupo de trabajo.}
	\label{fig:grupo_trabajo}
\end{figure}

Comentamos a continuación las acciones posibles a realizar en un grupo de trabajo \ref{fig:opciones_grupo_trabajo}:

\begin{figure}[H]
	\centering
	\includegraphics[width=0.3\textwidth]{imagenes/opciones_grupo_trabajo}
	\caption{Grupo de trabajo. Opciones de un grupo de trabajo.}
	\label{fig:opciones_grupo_trabajo}
\end{figure}

\begin{itemize}
	\item \textbf{Subir archivo al grupo:} seleccionamos un archivo en el dispositivo usado y lo añadimos a nuestro grupo. Todos los participantes tendrán los mismos permisos sobre él.
	\item \textbf{Crear carpeta:} añade un directorio al grupo de trabajo.
	\item \textbf{Añadir participante al grupo:} añade un participante al grupo de trabajo.
	\item \textbf{Ver participantes del grupo:} muestra una lista con los participantes del grupo de trabajo.
	\item \textbf{Salir del grupo de trabajo:} el usuario deja de ser un participante del grupo. Cuando el grupo queda sin participantes, se elimina automáticamente junto con todo su contenido.
\end{itemize}

\subsubsection{Opciones para los archivos dentro de un grupo de trabajo:}
\begin{itemize}
	\item \textbf{Descargar:} permite transferir el archivo desde la nube hasta el dispositivo usado.
	\item \textbf{Cambiar nombre:} renombrar un archivo.
	\item \textbf{Copiar en:} realiza una copia de un archivo.
	\item \textbf{Mover a:} mueve el archivo a otro directorio.
	\item \textbf{Eliminar:} borra de la nube todos los datos relacionados con el archivo.
\end{itemize}

\subsubsection{Opciones para los directorios dentro de un grupo de trabajo:}
\begin{itemize}
	\item \textbf{Cambiar nombre:} renombrar un directorio.
	\item \textbf{Copiar en:} realiza una copia de un directorio.
	\item \textbf{Mover a:} mueve el directorio a otro directorio.
	\item \textbf{Eliminar:} borra de la nube el directorio y todo su contenido.
\end{itemize} 

\section{Opciones de usuario}
Si nos fijamos, en la esquina superior derecha nos aparece un icono con nuestra imagen de perfil y una flechita a su derecha. Pulsando sobre tal icono nos muestra otra lista de opciones que vemos en la figura \ref{fig:opciones_usuarios}.

\begin{figure}[H]
	\centering
	\includegraphics[width=1\textwidth]{imagenes/opciones_usuarios}
	\caption{Opciones de usuario.}
	\label{fig:opciones_usuarios}
\end{figure}

\subsection{Usuario}
El apartado \textbf{Usuario} muestra información referente al cliente \ref{fig:usuario}. Desde esta sección podremos consultar información sobre nuestra cuenta, cambiar nuestra imagen de perfil (haciendo click sobre la propia imagen) o eliminar la cuenta.

\begin{figure}[H]
	\centering
	\includegraphics[width=1\textwidth]{imagenes/usuario}
	\caption{Opciones de usuario. Usuario.}
	\label{fig:usuario}
\end{figure}

\subsection{Ayuda}
La página de \textbf{Ayuda} \ref{fig:ayuda} detalla las acciones posibles que puede realizar el usuario en cada apartado. Para elegir entre los distintos apartados desplegamos el botón \textbf{Ayuda} y seleccionamos el tema sobre el que queremos recibir información.

\begin{figure}[H]
	\centering
	\includegraphics[width=1\textwidth]{imagenes/ayuda}
	\caption{Opciones de usuario. Ayuda.}
	\label{fig:ayuda}
\end{figure}

\subsection{Salir}
Pulsando en \textbf{Salir} cerramos la sesión actual y volvemos a la página de login.
%
%%\chapter{Conclusiones y Trabajos Futuros}
%
%
%%\nocite{*}
%\bibliography{bibliografia/bibliografia}\addcontentsline{toc}{chapter}{Bibliografía}
%\bibliographystyle{miunsrturl}
%
%\appendix
%\input{apendices/manual_usuario/manual_usuario}
%%\input{apendices/paper/paper}
%\input{glosario/entradas_glosario}
% \addcontentsline{toc}{chapter}{Glosario}
% \printglossary
\chapter*{}
\thispagestyle{empty}


\newpage
\bibliography{citas} %archivo citas.bib que contiene las entradas 
\bibliographystyle{ieeetr} % hay varias formas de citar
\end{document}
