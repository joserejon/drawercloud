\chapter{Introducción}

El objetivo de este proyecto es el desarrollo de una aplicación web que proporcione al usuario la posibilidad de almacenar sus archivos en la nube. La aplicación web cuenta con opciones como el almacenamiento en un lugar seguro, la organización por carpetas, la posibilidad de crear grupos de trabajo donde compartir los archivos con compañeros, entre otras más.

\section{Motivación}

Unos de los problemas que surgen en el uso de nuestros dispositivos es el almacenamiento. Es cierto que el precio del \textbf{gigabyte} es cada vez más barato e incluso más rápido (como por ejemplo en los discos duros SSD, que están dotados de un gran incremento de velocidad, además de consumir menos energía) pero, aunque en un ordenador el almacenamiento puede no ser un problema, si que lo es si hablamos de dispositivos móviles como smartphones o tablets. A día de hoy, encontramos dispositivos que no permiten la insercción de tarjetas de almacenamiento flash, lo cual nos obliga a conformarnos con el espacio que el fabricante pone a nuestra disposición (o a comprar el modelo con más capacidad, pero obviamente más caro). \textbf{drawercloud} viene a poner solución a este problema, ofreciendo un espacio personal en un servidor web dedicado para dicha aplicación, de modo que nuestros datos permanecen en tal servidor sin ocupar memoria en nuestros dispositivos. \\

¿Significa esto que solo es una solución para dispositivos que cuentan con un espacio muy limitado? No, ya que otro problema que puede surgirnos con mucha facilidad es la pérdida de información por diversas causas (disco duro roto, pérdida del dispositivo, virus, etc) y para ello este proyecto también ofrece una solución a dicho problema. Podemos almacenar nuestros archivos tanto en el disco duro propio del dispositivo en cuestión, como en el espacio ofrecido en el servidor web, de modo que si nos vemos afectados por alguno de los motivos mencionados, o bien por otros que afectan a la desaparición de nuestra información, tengamos una copia en otro lugar y no perdamos nuestros archivos. \\

Otro caso en el que esta aplicación resulta interesante es que podemos disponer de nuestros datos en cualquier parte. A modo de ejemplo, podemos pensar en un archivo que estamos modificando en casa y en nuestro puesto de trabajo. Almacenando dicho archivo en la nube nos tenemos que olvidar de mover la información a través de, por ejemplo, un pen drive o de tener una copia en casa y otra en la oficina y andar actualizando ambas cada vez que queramos trabajar. Simplemente entraremos con nuestra cuenta a \textbf{drawercloud} desde cualquier dispositivo y tendremos acceso al archivo en cuestión. Además, proporciona un método para compartir archivos y crear grupos de trabajo, de modo que si sobre ese archivo están trabajando, por ejemplo, 3 personas más, podrán ver las modificaciones que hemos realizado directamente sobre el archivo alojado en el servidor web.
