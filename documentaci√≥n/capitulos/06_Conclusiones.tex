\chapter{Conclusiones y trabajos futuros}

\section{Conclusiones}
Tras la finalización del proyecto se ha obtenido un prototipo funcional de la aplicación web, que cumple con casi todas las expectativas y los objetivos marcados desde el principio, además de una completa memoria del proyecto. \\
 
Gracias a las pruebas realizadas con usuarios podemos determinar que...... \\

En definitiva tenemos que \textbf{drawercloud} cumple con los requisitos para efectuar su función principal de alojar archivos en la nube, a excepción del espacio disponible de almacenamiento actual que es insuficiente para una aplicación de estas características. Esto se debe a que el espacio que nos ofrece Amazon es solamente de 2GB, debiéndose esto a que se trata de la versión gratuita. No obstante esto tendría solución comprando más espacio de almacenamiento y ofreciendo a los clientes distintas versiones, las cuales serían de pago para grandes cantidades de espacio y una gratuita que ofrecería menos capacidad. \\

\subsection{Conocimientos adquiridos}
Gracias al análisis y al desarrollo de este proyecto se han adquirido conocimientos en diversos ámbitos del desarrollo de software, en su mayoría en el sector web. Entre los conocimientos adquiridos podemos nombrar: 

\begin{itemize}
	\item \textbf{Python:}
	\item \textbf{Django:}
	\item \textbf{Arquitectura MVC:}
	\item \textbf{Tecnologías web:}
	\item \textbf{Pruebas con usuarios:}
	\item \textbf{Desarrollo de documentación:}
\end{itemize}

\section{Trabajos futuros}