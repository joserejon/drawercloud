\chapter{Pruebas}

Se han realizado distintas pruebas para estudiar el comportamiendo de la aplicación web (dichas pruebas se realizaron con la última versión de la aplicación). Para realizar las pruebas se ha usado la ayuda de un grupo de 10 personas, las cuales han estado interactuando con la aplicación durante unos 10 minutos aproximadamente. Los resultados han sido los siguientes:

\section{Resultados de las pruebas con usuarios}
La prueba se iniciaba realizando dos preguntas al usuario:

\begin{itemize}
	\item ¿Sabes lo que es la nube?
	\item ¿Has usado alguna aplicación que use esta tecnología?
\end{itemize}

El 100\% de los usuarios conocía que era la nube y solo tres de ellos no habían usado nunca este tipo de aplicación. \\


Una vez se realizaban las preguntas se dejaba al usuario usar la aplicación web solo, de modo que fuera él mismo quien resolviera sus dudas. Durante la prueba se hizo uso de todas las acciones principales, como subir archivos, organizarlos en directorios, consultar el apartado multimedia, compartir archivos, agregar a favoritos y trabajar con los grupos de trabajo. \\

\subsubsection{¿Qué podemos destacar de las pruebas?}

\begin{itemize}{}
	\item Se puede determinar que se usa una interfaz intuitiva ya que los usuarios se desenvolvieron rápida y correctamente visitando cada página y haciendo uso de las funciones básicas. En el caso de los usuarios que no habían usado nunca una aplicación de este tipo no hubo problemas destacables tampoco.
	\item El uso del botón derecho por encima del botón "Más" para mostrar las opciones de directorios/archivos. Apenas se llegó a usar dicho botón.
	\item El botón con la imagen de perfil de usuario que muestra las opciones de Usuario, Ayuda y Salir pasó desapercibido para un total de 7 usuarios.
	\item 4 usuarios intentaron seleccionar archivos, pero esta funcionalidad no está disponible.
	\item La acción de arrastrar para mover un archivo a una carpeta la intentaron 9 personas, pero esta opción tampoco se desarrolló en esta versión del proyecto.
	\item 3 usuarios intentaron subir varios archivos a la vez y al igual que antes, esta versión no cuenta con tal opción.
\end{itemize}

En resumen podemos concluir que la aplicación tiene una interfaz intuitiva y que cubre las necesidades básicas para desempeñar su función de almacenador archivos en la nube. Se deberían corregir sin embargo los problemas más repetidos.